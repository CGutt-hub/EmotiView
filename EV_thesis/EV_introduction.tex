\section{Introduction}

The human experience is profoundly shaped by emotions, which orchestrate adaptive responses to environmental challenges and opportunities. Not purely mental events, emotions manifest as complex, integrated states involving subjective feelings, cognitive appraisals, behavioral expressions, and significant physiological changes mediated by the central and \gls{ANS} \parencite{barrettHandbookEmotions2016, kreibigAutonomicNervousSystem2010}. Understanding the dynamic interplay between neural processing and autonomic physiology is fundamental to emotion science \parencite{critchleyNeuralMechanismsAutonomic2005}. This brain-body interaction is not merely epiphenomenal; the perception and regulation of bodily states (interoception) actively contribute to emotional feeling and decision-making \parencite{antoniodamasioDescartesErrorEmotion2005}. Furthermore, disruptions in this neuro-autonomic dialogue, such as altered heart rate variability or electrodermal hypo/hyper-reactivity, are increasingly recognized as core features of affective disorders like anxiety and depression, underscoring the clinical relevance of investigating these mechanisms \parencite{thayerHeartRateVariability2009, beauchaineVagalToneDevelopment2001}.\\

Within the brain, the \gls{PFC} is critical for orchestrating emotional responses, integrating sensory information with internal goals and modulating subcortical affective circuits \parencite{fusterPrefrontalCortex2008}. A prominent theoretical framework focuses on hemispheric asymmetry in \gls{PFC} activity. The motivational direction model posits that relatively greater left \gls{PFC} activation supports approach-related motivation and positive affect, while relatively greater right \gls{PFC} activation underlies withdrawal-related motivation and negative affect \parencite{davidsonWhatDoesPrefrontal2004, harmon-jonesAngerFrontalBrain1996}. This asymmetry is not static but dynamically modulated by emotional context and individual predispositions, such as trait approach/avoidance tendencies captured by the \gls{BIS} and \gls{BAS} scales \parencite{carverBehavioralInhibitionBehavioral1994, suttonPrefrontalBrainAsymmetry1997, rodriguesMindMovementFrontal2018}. The effectiveness of emotional induction methods can also influence the observed asymmetry, suggesting that more engaging stimuli may elicit stronger, more differentiated neural responses \parencite{rodriguesMethodsMatterExamination2021}.\\

% Intro Para 3 (HRV/EDA) - Condensed RMSSD/EDA description
While the \gls{PFC} orchestrates central processing, the emotional response simultaneously engages the body via the \gls{ANS}, adjusting physiology to meet situational demands. Different physiological signals offer complementary views of autonomic activity. \gls{HRV}, the analysis of beat-to-beat fluctuations in heart rate derived from \gls{ECG}, provides a non-invasive window into \gls{ANS} function \parencite{malikHeartRateVariability1996, berntsonHeartRateVariability1997}. Short-term metrics like \gls{RMSSD} primarily reflect parasympathetic (vagal) control \parencite{malikHeartRateVariability1996, shafferOverviewHeartRate2017}. Higher vagally-mediated \gls{HRV} (indexed by \gls{RMSSD}) generally reflects greater autonomic flexibility and adaptive capacity, linked to effective emotional regulation, while reduced \gls{HRV} is linked to stress, psychopathology, and impaired executive function \parencite{thayerRelationshipAutonomicImbalance2010}. In contrast, sympathetic activity can be captured using \gls{EDA} (\gls{GSR}), which measures changes in skin conductance driven by sympathetically-innervated sweat glands \parencite{dawsonElectrodermalSystem2007, boucseinElectrodermalActivity2012}. Phasic increases (\gls{SCR}) reflect event-related sympathetic bursts, while the tonic \gls{SCL} indicates general arousal \parencite{boucseinElectrodermalActivity2012}. Capturing both parasympathetic (\gls{RMSSD}) and sympathetic (\gls{EDA}) influences allows comprehensive assessment of the body's response during emotion.\\

The perception of emotion itself is also debated. While traditional appraisal theories emphasize cognitive interpretation as the primary determinant of emotional experience \parencite{schererAppraisalProcessesEmotion2001}, alternative perspectives like the direct perception account suggest a more immediate recognition of emotional significance based on characteristic patterns of features, akin to perceptual object recognition \parencite{newenEmotionRecognitionPattern2015}. This view aligns well with theories of embodied cognition, which posit that cognitive processes, including emotional understanding, are grounded in the body's sensorimotor and physiological states \parencite{niedenthalEmbodimentEmotionConcepts2009}. From this perspective, the coordinated activity between neural circuits (like the \gls{PFC}) and autonomic responses (reflected in \gls{RMSSD} and \gls{EDA}) is not just a correlate but an integral part of the emotional percept and experience \parencite{critchleyNeuralMechanismsAutonomic2005}. Understanding how these systems dynamically couple (synchronize) over time is key to understanding the emergence of subjective feeling states. \\

Investigating this embodied perspective, where coordinated brain-body activity is integral to emotion, necessitates analytical approaches capturing the temporal coordination (synchrony) between brain and body signals. Neural-autonomic synchrony reflects the functional coupling between central neural oscillations and peripheral physiological rhythms \parencite{thayerHeartRateVariability2009}. Quantifying this synchrony reveals how effectively these systems interact during emotional challenges. The \gls{PLV} is a robust method for this purpose \parencite{lachauxMeasuringPhaseSynchrony1999}. \gls{PLV} measures the consistency of the phase difference between two time series, irrespective of their amplitudes. A high \gls{PLV} indicates consistent phase alignment, suggesting a stable functional interaction. We will calculate \gls{PLV} between the \gls{EEG} signal from the \gls{PFC} and two different autonomic signals: separately with a continuous \gls{HRV}-derived signal (reflecting parasympathetic influence) and with a continuous \gls{EDA}-derived signal (reflecting sympathetic influence). Compared to coherence measures, which are sensitive to both phase and amplitude, \gls{PLV}'s focus on phase makes it particularly adept at detecting transient, potentially non-linear synchronization patterns characteristic of dynamic biological systems responding to stimuli \parencite{cohenAnalyzingNeuralTime2014, valenzaRevealingRealTimeEmotional2014}. The proposed theoretical integration is depicted in \autoref{fig:neuralmodel}.\\

The present study aims to elucidate neural-autonomic synchrony during the conscious processing of distinct emotional states elicited by validated affective stimuli. We adopt a multimodal approach, integrating high-temporal-resolution \gls{EEG} to capture rapid prefrontal neural dynamics, \gls{ECG} to derive measures of primarily parasympathetic autonomic control (\gls{RMSSD}), \gls{EDA} to reflect sympathetic sudomotor control, and \gls{fNIRS} for complementary spatial localization \parencite{scholkmannReviewContinuousWave2014, pintiCurrentStatusIssues2019}. This combination allows investigating network dynamics while focusing on \gls{EEG}'s temporal resolution for synchrony, \gls{ECG}/\gls{EDA} for specific autonomic branches, and \gls{fNIRS} to localize activity in key prefrontal (emotion regulation) and parietal (attention, bodily signal integration) regions \parencite{pessoaRelationshipEmotionCognition2008, critchleyNeuralMechanismsAutonomic2005}. While technically demanding, this integrated approach is necessary to investigate how precisely timed brain-body coupling relates to activity within specific neural substrates. We acknowledge that linking the hemodynamic response measured by \gls{fNIRS} to the specific neural activity whose phase is relevant for autonomic coupling relies on the assumption that significant hemodynamic changes reflect underlying neural engagement relevant to the synchrony analysis, despite the complex and indirect nature of neurovascular coupling \parencite{logothetisInterpretingBOLDSignal2004}. The \gls{fNIRS} data, therefore, serve not just as post-hoc context, but as a means to spatially constrain and interpret the functional significance of observed \gls{EEG}-based synchrony patterns, as detailed in the analysis plan. We will utilize standardized, emotionally evocative video clips (positive, negative, neutral) from the E-MOVIE database, validated by \textcite{maffeiEMOVIEExperimentalMOVies2019} for their efficacy in inducing targeted affective states. By calculating \gls{PLV} between prefrontal \gls{EEG} oscillations (specifically in the alpha [8-13 Hz] and beta [13-30 Hz] bands, which are broadly implicated in cortical processing, attention, and potentially emotional regulation \parencite{klimeschEEGAlphaTheta1999, koenigMillisecondMillisecondYear2002, allenIssuesAssumptionsRoad2004}) and relevant autonomic signals (including continuous signals derived from \gls{HRV} data reflecting vagal activity for brain-heart coupling and continuous phasic \gls{EDA}-derived signals for brain-sudomotor sympathetic coupling), we will investigate how brain-body interactions involving both \gls{ANS} branches vary with emotional content and intensity.\\

% --- Figure Environment for Conceptual Model ---
\begin{figure}[htbp] % h=here, t=top, b=bottom, p=page of floats; ! forces placement
    \centering
    \begin{tikzpicture}[
        scale=0.95, transform shape, % Scales the drawing slightly
        node distance=1cm, % Default distance between nodes
        every node/.style={text width=3cm, align=center}, % Default style for all nodes
        mynode/.style={rectangle, draw}, % Custom style for rectangular nodes with border
        myarrow/.style={-Latex} % Custom style for arrows
    ]
    % --- TikZ Nodes (positioning and content) ---
    \node (A) [mynode] at (0,0) {Emotional Stimulus};
    \node (B) [mynode, below=of A, text width=4cm] {Cognitive Appraisal};
    \node (C) [mynode, left=of B, text width=4cm] {Context and prior knowledge};
    \node (D) [below=of B]{}; % Invisible node for positioning help
    \node (E) [mynode, right=of D, text width=4cm] {Prefrontal Asymmetry};
    \node (F) [mynode, left=of D, text width=4cm] {Physiological response};
    \node (G) [mynode, below=of D, text width=4cm] {Neural-Autonomic Synchrony};
    \node (H) [mynode, below=of G, text width=4cm] {Emotional Experience and Perception};
    % --- TikZ Connections (drawing arrows between nodes) ---
    \draw[myarrow] (A) -- (B);
    \draw[myarrow] (C) -- (B);
    \draw[myarrow] (B) -- (E);
    \draw[myarrow] (B) -- (F);
    \draw[myarrow] (E) -- (G);
    \draw[myarrow] (F) -- (G);
    \draw[myarrow] (G) -- (H);
    \end{tikzpicture}
    \caption{Neural-Autonomic Synchrony in Emotional Experience.}
    \label{fig:neuralmodel} % Label for referencing the figure (\autoref{fig:neuralmodel})
    % \vspace{0.5em} % Optional vertical space after figure
    \parbox{0.8\textwidth}{\scriptsize This diagram illustrates the proposed model of emotional processing, highlighting the interplay between neural and physiological components. An emotional stimulus initiates cognitive appraisal, influenced by contextual factors. Simultaneously, the stimulus elicits a physiological response. Cognitive appraisal leads to prefrontal asymmetry, a key neural component, which interacts with the physiological response through neural-autonomic synchrony. This synchronized activity culminates in the emotional experience.} % Detailed description below caption
\end{figure}

% --- Hypotheses ---
Based on these considerations, we propose the following hypotheses, organized into distinct \gls{WP}:

\begin{enumerate}[label=(\alph*)]
    \item \textbf{Neural-Autonomic Synchrony will be significantly enhanced during the processing of emotional stimuli.} We hypothesize that neural-autonomic synchrony (\gls{PLV}) will be significantly enhanced during the processing of both positive and negative emotional stimuli compared to neutral stimuli. We will investigate this in \gls{WP}1: \emph{Emotional Modulation of Synchrony}, using \gls{PLV} calculated between prefrontal \gls{EEG} and both continuous \gls{HRV}-derived signals (reflecting brain-heart parasympathetic coupling) and \gls{EDA}-derived signals (reflecting brain-sudomotor sympathetic coupling). Rationale: Emotional engagement necessitates greater integration between central command centers and peripheral effectors (cardiac and sudomotor) to prepare for potential action and adapt internal milieu; this heightened integration is expected to manifest as stronger phase coupling involving both \gls{ANS} branches \parencite{critchleyNeuralMechanismsAutonomic2005}. While we predict overall enhancement, we will also exploratorily examine whether the nature or dominant branch of synchrony (e.g., \gls{EEG}-\gls{HRV} vs. \gls{EEG}-\gls{EDA}) differs between positive and negative valence conditions.
    \item \textbf{The magnitude of neural-autonomic synchrony is positively correlated with subjective ratings of emotional arousal.} We hypothesize a positive correlation between the magnitude of neural-autonomic synchrony and subjective ratings of emotional arousal (via the \gls{SAM}) specifically during the positive and negative conditions. This correlation will be tested in \gls{WP}2: \emph{Synchrony and Subject Arousal} for both \gls{PLV} using continuous \gls{HRV}-derived signals (parasympathetic coupling) and \gls{PLV} using \gls{EDA}-derived signals (sympathetic coupling). Rationale: Higher subjective arousal reflects a more intense emotional state, likely involving stronger reciprocal signaling and tighter coupling between the brain's assessment and the body's physiological mobilization. This mobilization involves coordinated adjustments in cardiovascular and sudomotor control (potentially reflected in \gls{EEG}-\gls{HRV} or \gls{EEG}-\gls{EDA} synchrony), leading to more consistent phase relationships between central and peripheral systems \parencite{valenzaRevealingRealTimeEmotional2014, boucseinElectrodermalActivity2012}.
    \item \textbf{Individual differences in baseline parasympathetic regulation is positively assiciated with the degree of neural-autonomic synchrony.} We hypothesize that individual differences in baseline parasympathetic regulation, indexed by resting-state \gls{RMSSD}, will be positively associated with the degree of neural-autonomic synchrony exhibited during the processing of negative emotional stimuli. In \gls{WP}3: \emph{Baseline Vagal Tone and Task-Related Synchrony} PLV will be calculated specifically between \gls{EEG} and continuous \gls{HRV}-derived signals. Rationale: Individuals with higher baseline \gls{RMSSD}, indicative of greater vagal regulatory capacity \parencite{thayerRelationshipAutonomicImbalance2010, shafferOverviewHeartRate2017}, may be better equipped to dynamically coordinate neural and cardiac activity when facing demanding or threatening situations, resulting in more robust task-related brain-heart phase synchrony.
    \item \textbf{The direction of prefrontal cortical asymmetry is differentially associated with the strength of phase synchrony.} We hypothesize that the direction of prefrontal cortical asymmetry, indexed by alpha band [8-13 Hz] power differences (e.g., F3 vs. F4), may be differentially associated with the strength of phase synchrony involving distinct autonomic branches. Specifically, we will exploratorily investigate in \gls{WP}4: \emph{Frontal Asymmetry and Branch-Specific Synchrony}, whether greater relative right \gls{PFC} activation (associated with withdrawal/inhibition) correlates with stronger \gls{EEG}-\gls{HRV} \gls{PLV} (parasympathetic coupling), and whether greater relative left \gls{PFC} activation (associated with approach/action) correlates with stronger \gls{EEG}-\gls{EDA} \gls{PLV} (sympathetic coupling), particularly during emotional conditions. Rationale: This hypothesis explores whether the functional specialization suggested by the motivational direction model \parencite{davidsonWhatDoesPrefrontal2004, harmon-jonesAngerFrontalBrain1996}, typically assessed via power measures, extends to the domain of phase coupling with the autonomic branch potentially most relevant to that motivational state. We acknowledge this is an exploratory step, bridging asymmetry (power) and synchrony (phase) concepts, aimed at generating hypotheses about potential underlying mechanisms linking cortical motivational states to specific patterns of brain-body interaction.
\end{enumerate}