Emotional states are fundamentally embodied, emerging from the dynamic interplay between central neural processing and peripheral physiological adjustments orchestrated by the autonomic nervous system (ANS). While ANS outputs like heart rate variability (HRV) and electrodermal activity (EDA) reflect emotional arousal and valence, understanding the precise temporal coordination between brain activity and these peripheral signals is crucial for elucidating brain-body interactions. This study investigates neural-autonomic phase synchrony during the conscious processing of distinct emotional states (positive, negative, neutral) by quantifying the temporal alignment between cortical and physiological rhythms.
    
We employ a multimodal approach, simultaneously recording high-temporal-resolution electroencephalography (EEG), electrocardiography (ECG) for HRV analysis (specifically Root Mean Square of Successive Differences, RMSSD), EDA, and functional near-infrared spectroscopy (fNIRS) while participants view validated emotional video clips. Our primary analysis quantifies the Phase Locking Value (PLV) between frontal EEG oscillations (Alpha, Beta bands) and continuous signals derived from HRV (reflecting parasympathetic influence) and phasic EDA (reflecting sympathetic influence). EEG channel selection for PLV analysis was informed by task-related hemodynamic activity measured via fNIRS to focus on functionally relevant cortical areas.
    
We hypothesize that PLV, indicating brain-body temporal integration, will be significantly modulated by emotional content compared to neutral conditions. We further expect synchrony strength to correlate with subjective arousal ratings. By examining the phase synchrony between brain signals and ANS-mediated physiological outputs, this research provides novel insights into the dynamic, embodied mechanisms underlying emotional experience. Understanding this temporal binding is critical for models of psychophysiological function and may inform assessments of cognitive load or stress regulation capacity, potentially impacting performance monitoring and optimization in demanding operational environments.